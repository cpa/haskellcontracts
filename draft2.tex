\documentclass{article}
% Test comment

\usepackage{stmaryrd}
\usepackage{amsmath}
\usepackage{fullpage}

\begin{document}

\newcommand{\etrans}[1]{\mathcal{E} \llbracket #1 \rrbracket}
\newcommand{\dtrans}[1]{\mathcal{D} \llbracket #1 \rrbracket}
\newcommand{\strans}[1]{\mathcal{S} \llbracket #1 \rrbracket}
\newcommand{\trans}[1]{\llbracket #1 \rrbracket}

\newcommand{\unr}{\texttt{UNR}}
\newcommand{\bad}{\texttt{BAD}}
\newcommand{\any}{\texttt{Any}}
\newcommand{\ok}{\texttt{Ok}}

\thispagestyle{empty}
\section{Grammars}
\subsection{$\lambda$-lifted haskell subset}
\begin{eqnarray*}
  u,e &:=& x \mid f \mid e~e \mid D \mid \bad \mid n & (Expression)\\
  p &:=& \Delta_1 \dots \Delta_n & (Program)\\
  \Delta &:=& d \mid transp (d,c) \mid opaque (d,c) & (Defintion~and~contract)\\
  d &:=& f~x_1 \dots x_n = e \mid f~x_1 \dots x_n = \mbox{ case } e \mbox{ of } [(pat_i,e_i)] & (Definition)\\
  pat &:=&D~x_1 \dots x_n & (Pattern)\\
\end{eqnarray*}

\subsection{FOL}
\begin{eqnarray*}
  t &:=& x \mid \mbox{app}(t_1,t_2) \mid k & (Term)\\
  k &:=& D \mid f \mid ? \mid \bad \mid \unr & (Constant)\\
  \phi &:=& \forall x.\phi \mid \phi \to \phi \mid \lnot \phi \mid \phi \lor \phi \mid \phi \land \phi \mid true \mid t=t \mid \mbox{CF}(t)& (Formula)\\
\end{eqnarray*}

\subsection{Contracts}
\begin{eqnarray*}
  c &:=& x:c_1 \to c_2\\
  &\mid& (c_1,c_2) \\
  &\mid& \{ x \mid e \}\\
  &\mid& \any \\
\end{eqnarray*}
We can consider CF as a user contract.

\section{Translation}
We define several translations: $\etrans{},\dtrans{},\strans{},\trans{}$.
\begin{eqnarray*}
  \etrans{} &::& Expression \to Term\\
  \dtrans{} &::& Definition \to FOF\\
  \strans{} &::& Expression \to Contract \to FOF\\
  \trans {} &::& Definition \to Contract \to FOF\\
\end{eqnarray*}

\subsection{$\etrans{}$}
$\etrans{e}$ is a term. The translation is direct.

\subsection{$\dtrans{}$}
$\dtrans{d}$ is a first-order formula.

\begin{eqnarray*}
\dtrans{f~x_1 \dots x_n = e} &=& \forall x_1 \dots x_n.\etrans{f~x_1 \dots x_n} = \etrans{e} &\\
\dtrans{f~x_1 \dots x_n = \mbox{case } e \mbox{ of } [D_i~\overline{z} \mapsto e_i]} &=& \forall x_1 \dots x_n~(\bigwedge_i (\forall \overline{z}~\etrans{e} = \etrans{D_i~\overline{z}} \to \etrans{f~x_1 \dots x_n} = \etrans{e_i}) &\\
&& \land \etrans{e} = \bad \to \etrans{f~x_1 \dots x_n} = \bad)&\\ 
&& \land \etrans{f~x_1 \dots x_n} = \unr \bigvee_i (\texttt{HD}(e) = D_i) & (\varphi)\\
\end{eqnarray*}

\subsection{$\strans{}$}
$\strans{e \in c}$ is a first-order formula.
\begin{eqnarray}
\strans{e \in \any} &=& true\\
\strans{e \in \{x \mid u \}} &=& \unr \lor (\mbox{CF}(\etrans{e}) \land \etrans{u[e/x]} \neq \bad \land \etrans{u[e/x]} \neq False)\\
\strans{e \in x:c_1 \to c_2}  &=& \forall x_1. \strans{x_1 \in c_1} \to \strans{e~x_1 \in c_2[x_1/x]}\\
\end{eqnarray}
$False$ is a data constructor here.

Remark: we follow the semantics of the POPL paper but it's a bit restrictive.
e.g. in equation 2 we could use the alternate semantics (namely B1 in the POPL paper) : 
$$\strans{e \in \{x \mid u \}} = \unr \lor (\etrans{u[e/x]} \neq \bad \land \etrans{u[e/x]} \neq False)$$

\subsection{$\trans{}$}
It's the final translation, which takes a function definition and its contract and returns a first-order formula\\

\thispagestyle{empty}

\begin{eqnarray*}
\trans{opaque(f~x_1 \dots x_n = e,c)} &=& \dtrans{f~x_1 \dots x_n = e[f_p/f]} \land \strans{f \in c} \land \strans{f_p \in c}\\
\trans{transp(f~x_1 \dots x_n = e,c)} &=& \dtrans{f~x_1 \dots x_n = e[f_p/f]} \land \strans{f \in c} \land \strans{f_p \in c}\\
\trans{f~x_1 \dots x_n = e} &=& \trans{opaque(f~x_1\dots x_n) = e,Ok \to \dots \to Ok)}\\
\end{eqnarray*}

We'd like a typical contract-checking session to go like this:
\begin{enumerate}
\item Start with an empty theory $T$.
\item Let $f~x_1 \dots x_n = e \in c$ be an opaque function definition to check wrt contract $c$. Check (with equinox) the consistency of the theory $T' = T \cup \trans{f~x_1 \dots x_n = e \in c}$
\item If $T'$ is consistent then let $T = \strans{f \in c} \cup T$ and go to 2. with the next function definition; otherwise give a counter-example and ask the user for refinement of the contracts and/or lemmas(?)
\end{enumerate}

\section{Questions}

Here are some open issues, design choices and equinox-related questions.

\begin{enumerate}
\item Nested implications may lead to existential quantification, is it troublesome in equinox? (altough we don't have this case here)
\item More generally, does equinox accept any FOF the grammar here defines?
\item In the section $\dtrans{}$, we believe replacing $\varphi$ by $\lor \trans{f x_1 \dots x_n} = \unr$ is equivalent. What's better for equinox?
\item Is the session stuff realistic? It looks like it can have a quadratic behaviour but maybe with the right API it's ok?
\end{enumerate}

\thispagestyle{empty}
\end{document}

